\subsection{Cross-platform build}
\subsubsection{Generate build setup}
Every platform has their own way of compiling source code into binaries. On Unix systems, gcc and makefiles is the common option while on Windows systems most code is compiled with Microsoft Visual Studio. CMake is a cross-platform build system by Kitware[ref]. CMake controls the software compilation step using platform independent configuration files. Option variables and cached string values can be defined in the configuration files and later on be modified at either the command line version of cmake or the gui version. This makes CMake a very powerful tool to setup customizable builds. For example, in gpuip, one can easily disable the build of the python bindings if it is not needed. Another case could be if the compiling system does not have a NVIDIA GPU and want to build gpuip without CUDA support. Once all options are set, CMake generates either Unix Makefiles, Microsoft Visual Studio or other build setups that are already configured. This means all the include paths for header files have been set and linking to other libraries is taken care of.

\subsubsection{Library dependencies}
Gpuip and especially its python bindings part depends on other open source libraries. When the compiler is invoked, information about where these libraries are located has to be passed. These locations can be vary a lot in different setups and it is hard to come up with a solution that is going to work nicely across all platforms. Luckily, CMake has a nice feature called {\tt FindPackage} where it is possible to register scripts to find external libraries. The most common libraries and their {\tt FindPackage} script are shipped with CMake. Some of the libraries used by gpuip were not reconized by {\tt FindPackage} in CMake and scripts for finding them were added.
\newline

It can be annoying to prepare all the prequisites building a library that depends on a lot of other libraries. To simplify this step, gpuip tries to make the build process as smooth and out of the box as possible. If a third party library is missing and it is an open source library, it will try to download the missing library at compile time and build it. This can be done through the {\tt ExternalProject\_Add} feature in CMake where one specify the path to the git or svn repository where the open source code exists. It is also possible to specify specific configure, build and install command if the open source library does not use CMake as build system.

\subsubsection{Regression testing}
CMake comes with ctest, which is tool that can be used for testing the code after building it. Gpuip has three differnet tests: One testing the standard API calls in C++, One testing the standard API calls in the python bindings and one that compares performance of gpuip vs cpu implementations (both single and multi-threaded).

\subsubsection{Documentation}
An API documentation is generated at build time (if the option is enabled) with the help of Doxygen[ref]. Doxygen reads the comments of the header files and generates an html and reader-friendly version to publish online. 

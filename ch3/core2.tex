One of the goals when designing \emph{gpuip} is to be an easy-to-use library. {\tt ImageProcessor} is the superclass and GLSL, CUDA and OpenCL would implemented as subclasses. The most common approach in object-oriented programming is to put subclasses in different header and source files. To use a certain implementation later on, one would have to include the header file of the implementation and create the subclass object itself (but store it as an pointer to the superclass for flexbility between different implementations. A typical user-case can be seen in Code \ref{includeex}.
\renewcommand{\lstlistingname}{Code}
\begin{lstlisting}[caption= Example of common use of object-oriented programming, label=includeex]
#include <gpuip-cuda.h>
#include <gpuip-glsl.h>
#include <gpuip-opencl.h>

void some_function(string environment)
{
  ImageProcessor ip;
  if(environment == 'cuda') {
    ip = new gpuip::CUDA();
  } else if(environment == 'glsl') {
    ip = new gpuip::OpenCL();
  } else if(environment == 'opencl') {
    ip = new gpuip::GLSL();
  }

  ...
}
\end{lstlisting}

Designing gpuip like the example in Code \ref{includeex} would make the public interface less clean, involve different header files and could potentially motivate the use of pointer casting which is not ideal[ref]. Instead, the public interface of gpuip consinsts of only one header file and the only exposed class is the {\tt ImageProcessor} class. Under the hood, there are still going to be subclasses, but the the developer using the library will not notice this. This enforces that all implementations always use the same function calls and work the same way. The use of factory functions are used once again, this time static as they are placed within the class itself.
\newline
\begin{lstlisting}[caption= {\tt ImageProcessor} static API, label=ipapi]
enum Environment { OPENCL, CUDA, GLSL };
class ImageProcessor {
    ... rest of ImageProccessor definitions ...

  static ImageProcessor::Ptr Create(Environment env);
  static bool CanCreate(Environemt env);
};
\end{lstlisting}

To make this work, we include all the implemented subclasses in the {\tt ImageProcessor} source file. An example of this can be seen in Code \ref{forw}
\newline
\begin{lstlisting}[caption= {\tt ImageProcessor::Create} spawns instances of subclasses to itself, label=ipapi]
// gpuip.cpp
#include <gpuip-cuda.h> 
static ImageProcessor::Ptr Create(Environment env)
{
  if (env == CUDA) {
    return ImageProcessor::Ptr(new CUDAImpl());
  }
  ...
}
\end{lstlisting}

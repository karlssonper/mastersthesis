\subsection{Python bindings}

Boost python [ref] is used to make the C++ code accessible in a python environment. When copying data from and to the GPU, one has to pass a void pointer in the gpuip C++ API. The concept of pointers does not exist in python. Instead, in the python bindings, the CPU data is attached to the buffer itself using a numpy [array]. All the data transfers between GPU buffers have to go through the numpy array.
\newline

A common task in image processing is to read image data from disk and later on write to disk once the processing is done. To simpify this step, read and write functions are included in the python bindings. Depending on the per element data in the numpy array, different file formats are available. For {\tt half} and {\tt float} precision, the target format is OpenEXR by ILM[ref]. When the data consists of unsigned bytes, the more common image formats png, jpeg, tiff and tga are available through the header-only library CImg[ref].  

\section{Conclusion}

The GPU environments available today are implemented differently but the main idea is the same in all of them:
\begin{enumerate}
\item Write GPU code and compile
\item Allocate GPU memory and transfer from/to CPU
\item Run GPU program
\end{enumerate}
To create a GPU framework for image processing, the public interface has to support these steps. Memory allocation and transferring is simplified if inputs and outputs to the algorithms are restricted to images only. Supporting other kinds of inputs and outputs is possible but makes it harder to generalize. Some functionality only exists in some GPU environments and it is up to the framework to decide if features that do not exists in all environments should be added or not. All three environments tested in this thesis had their pros and cons and which one to choose is case-dependent.
\newline

Using a GPU framework is worth it if an image processing algorithm requires per pixel computations and performance is important. The more computation required, the faster the GPU is compared to the CPU. The CPU only performs well in cases where there is little computation and the memory fetch pattern is linear.


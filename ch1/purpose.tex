\subsection{Purpose}

Most image processing tasks are well-suited for parallel computing. The average image consists of millions of individual pixels. This is a good case for GPU computing where a lot of the compute units can be utilized at the same time. To get started with GPU computing, one has to choose a parallel GPU environment to write the code in. There are different environments available, each with its own pros and cons. Which one to choose can sometimes be difficult to decide since it requires testing. In some cases, it might not be clear which environment is the best overall since they all have unique subfeatures.
\newline

GPU computing can be split into two steps. The first step is the configuration phase where the GPU environment is instantiated, memory is allocated on the GPU, data is copied to the GPU and the image processing code is compiled to GPU machine code. The configurations vary from environment to enviroment and it can be tedious to do this setup in every program with GPU computing. This motivates the use of a framework for efficient image processing on the GPU. This thesis will focus on implementing this framework, supporting the common GPU environments. It will try explore and hopefully answer the following questions:
\newline

\begin{itemize}
\item{What do the common GPU architectures have in common?}
\item{How do you generalize GPU computing for image processing?}
\item{Is it possible to write one functional framework although the environments and their architectures are different?}
\item{What restrictions have to be made?}
\item{Is it worth doing image processing on the GPU instead of the CPU?}
\end{itemize}

The second part of GPU computing is implementing the actual algorithms that run on the GPU. This part is hard to generalize since every GPU environment has its own coding language and special features. 
\newline

The work of this thesis will result in three different software components. They are the following:

\begin{enumerate}
\item{\bf{General backend framework}}
\begin{itemize}
\item{C++ library}
\item{Cross-platform}
\item{Easy to disable components from compilation}
\item{Possible to dynamically rebuild GPU code at runtime}
\item{Support python bindings}
\item{Unit tests for all implementations and cases}
\end{itemize}

\item{\bf{Graphical User Interface application}}
\begin{itemize}
\item{Python}
\item{Graphical display of the image outputs}
\item{Quickly change parameter values}
\item{Support for live coding}
\item{Save an algorithm setup for later use, including code and parameters}
\end{itemize}

\item{\bf{Command-line application}}
\begin{itemize}
\item{Python}
\item{Run the algorithm setup saved from the GUI-version}
\item{Change parameter values through command-line arguments.}
\end{itemize}
\end{enumerate}
